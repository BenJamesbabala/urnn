\documentclass{article} % For LaTeX2e
\usepackage{iclr2016_conference,times}
\usepackage{hyperref}
\usepackage{url}
\usepackage{graphicx}

\title{\scalebox{0.95}{Unitary Evolution Recurrent Neural Networks}}


\author{Martin Arjovsky \thanks{Indicates first authors. Ordering determined by coin flip.} \\
Universidad de Buenos Aires\\
\texttt{\{marjovsky\}@dc.uba.ar} \\
\And
Amar Shah$^*$ \\
Cambridge University \\
\texttt{\{as793\}@cam.ac.uk} \\
\AND
Yoshua Bengio \\
Universite de Montr\'eal, CIFAR Senior Fellow\\
\texttt{\{yoshua.bengio\}@gmail.com} \\
}

% The \author macro works with any number of authors. There are two commands
% used to separate the names and addresses of multiple authors: \And and \AND.
%
% Using \And between authors leaves it to \LaTeX{} to determine where to break
% the lines. Using \AND forces a linebreak at that point. So, if \LaTeX{}
% puts 3 of 4 authors names on the first line, and the last on the second
% line, try using \AND instead of \And before the third author name.

\newcommand{\fix}{\marginpar{FIX}}
\newcommand{\new}{\marginpar{NEW}}

%\iclrfinalcopy % Uncomment for camera-ready version

\begin{document}


\maketitle

\begin{abstract}
Recurrent neural networks (RNNs) are notoriously difficult to train. When the eigenvalues of the hidden to hidden weight matrix
deviate from absolute value 1, optimization becomes difficult due to the well studied issue of vanishing and exploding gradients, especially when trying to learn long-term dependencies.
To circumvent this problem, we propose a new architecture that learns a unitary weight matrix, with eigenvalues
of absolute value exactly 1. We construct an expressive unitary weight matrix by composing several structured matrices that act
as building blocks with parameters to be learned. Optimization of this parameterization becomes feasible only when considering hidden
states in the complex domain. We demonstrate the potential of this architecture by achieving state of the art in several hard tasks
involving very long-term dependencies.

\end{abstract}
\bibliography{iclr2016_conference}
\bibliographystyle{iclr2016_conference}

\end{document}
